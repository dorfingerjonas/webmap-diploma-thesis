\section{Frontend Framework Angular}
\setauthor{Jonas Dorfinger}
Angular ist eine von Google entwickelte Entwicklungsplattform um Frontends zu implementieren.
Als Plattform bietet Angular etliche Services und Features an, dazu gehören nicht nur die der Cross-Platform Support,
sondern auch eine built-in Geschwindigkeits- und Performanceoptimierung. Auch für Unternehmen, mit sehr hohen Nutzerzahlen
ist Angular ein gern gewähltes Tool. Ein großer Vorteil von Angular ist zudem, dass Angular unter eine Open-Source Lizenz veröffentlicht
ist und somit nicht von dem Angular-Google Team sondern auf von der Community betreut und erweitert wird.
Angular verfolgt bei der Implementierung das Konzept vom Model-View-Controller Desgin Pattern.

Angular zu verwenden ist eine Anforderung von der Betreuer Firma triply GmbH, es wird bereits für alle Frontend Projekte verwendet,
deshalb auch für dieses.

\subsubsection{Model-View-Controller Design Pattern}
\setauthor{Jonas Dorfinger}
Dieses Pattern wird oft für die Entwicklung von User Interfaces verwendet,
dabei wird die Logik in drei verschiedenen Elementen (Files) aufgeteilt. Das wird
gemacht um interne Abbildungen und Referenzen von Daten beziehungsweise Informationen
in einer Art und Weise aufzuteilen, wie der diese dem Nutzer angezeigt oder vom 
Nutzer akzeptiert werden. In Angular ergebn alle drei Files eine gemeinsame Component

\paragraph{Model}
Das Model File ist das Herzstück der Component, es beinhaltet alle Daten
Strukturen und den Aufbau, dabei werden direkt Daten, Logik und Regeln
der Component festgelegt und verwaltet.

\paragraph{View}
Die View beinhaltet alle Details zur Darstellungen der Daten und Informationen.
Es kann mehrere Views für die gleiche Information geben, zum Beispiel ein
Balkendiagram für den Manager oder eine Tabelle für die Buchhaltung.

\paragraph{Controller}
Der Controller 
Accepts input and converts it to commands for the model or view.


% https://blog.iandavis.com/2008/12/what-are-the-benefits-of-mvc/ besucht am 02.03.2022
% https://martinfowler.com/eaaDev/uiArchs.html besucht am 02.03.2022
% Image Resource: https://commons.wikimedia.org/wiki/File:MVC-Process.svg vom 02.03.2022


\subsubsection{Model}
\setauthor{Jonas Dorfinger}

\subsection{Alternativen für Angular}
\setauthor{Jonas Dorfinger}

\subsubsection{React}
\setauthor{Jonas Dorfinger}
ReactJS ist die größte Konkurrenz von Angular, mit einem vielfachen an Marktanteilen.
ReactJS ist eine JavaScript Bibliothek von Facebook. Im Gegensatz zu Angular, wird der DOM weg abstrahiert was ein einfacheres 
Programmiermodel und bessere Performance ermöglicht. Ein Alleinstellungsmerkmal ist dazu auch noch, dass React Server-Side Rendering 
unterstützt.

\subsubsection{Vue}
\setauthor{Jonas Dorfinger}
VueJS ist kaum verbreitet, und hat als größten Nachteil, dass es nur von einer Person regelmäßig erweitert und betreut wird, das kann man anhand der Contributer Activity von dem 

% https://angular.io/guide/what-is-angular besucht am 09.02.2022
% https://angular.io/features besucht am 09.02.2022
% https://github.com/vuejs/core/graphs/contributors besucht am 09.02.2022

Alle 3 Frameworks werden von den den größten IDEs unterstützt.

\section{Map Frameworks}
\setauthor{Jonas Dorfinger}
Um die GeoDaten im Browser darstellen zu können, wird eine Map benötigt. Es gibt dafür drei weit verbreitete Frameworks, MapBox, Leaflet und Google Maps.

\subsection{Mapbox}
\subsection{Leaflet}
\subsection{Google Maps}

\section{Static Site Generators}
\setauthor{Sebastian Scholl}
\subsection{Next.js}
\subsection{Jekyll}
\subsection{Scully}

\section{Deployment Pipeline}
\setauthor{Sebastian Scholl}
Nach dem Download des generierten Projekts sollte die Möglichkeit bestehen,
kleine Änderungen daran vorzunehmen und es dann schnell und einfach auf einen
Server zu deployen. Der Server ist dabei entweder eine Linux Virtual Machine
oder eine \textit{Firebase Hosting} Instanz.
Dazu gab es mehrere Ansätze: Ist das Projekt ein Git-Repository, kann eine
Automatisierungssoftware, wie Jenkins oder GitHub Actions verwendet werden,
die bei jedem \textit{Push}-Event das Projekt buildet und auf den Server
deployed. Die zweite Möglichkeit ist ein Node.js-Skript, das im Projekt
enthalten ist und von der Kommandozeile aus ausgeführt wird.
\subsection{Jenkins}
Jenkins ist ein Open-Source Automatisierungsserver. Das Projekt wird in
Java entwickelt und kann mit Hilfe von Plugins an spezifische Anforderungen
angepasst werden.
Der Server wird vor Allem zur Automatisierung von Aufgaben, wie das Builden,
Testen und Deployen von Softwareprojekten genutzt.
Jenkins muss selbst gehosted werden. Dazu wird ein offizielles Docker Image
im Docker Hub bereitgestellt.
Die Konfiguration einer Pipeline wird in einem \textit{Jenkinsfile} vorgenommen,
das sich im Git-Repository befindet.
Um die Dateien zur Virtual Machine zu senden, wurde das Plugin \textit{Publish Over SSH}
verwendet. Für das Deployen zu \textit{Firebase Hosting} steht kein Plugin zur
Verfügung. Stattdessen wurde auf die \textit{Firebase CLI} zurückgegriffen.

% https://wiki.eclipse.org/Jenkins#About_Jenkins, 13.3.
% https://github.com/jenkinsci/jenkins, 13.3.
% https://hub.docker.com/r/jenkins/jenkins, 13.3.
% https://www.jenkins.io/doc/book/pipeline/jenkinsfile/, 13.3.
% https://www.jenkins.io/doc/pipeline/steps/publish-over-ssh/, 13.3.
% https://firebase.google.com/docs/cli/, 13.3.
\subsection{GitHub Actions}

\section{Backend}
\setauthor{Sebastian Scholl}
\subsection{JavaScript}
\subsection{Typescript}

\section{Webserver}
\setauthor{Sebastian Scholl}

\section{Reverse Proxy}
\setauthor{Sebastian Scholl}

\section{Containerization}
\setauthor{Sebastian Scholl}
\subsection{Docker}
\subsection{Docker Compose}
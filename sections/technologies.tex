\section{Frontend Framework Angular}
\setauthor{Jonas Dorfinger}
Angular ist eine von Google entwickelte Entwicklungsplattform um Frontends zu implementieren.
Als Plattform bietet Angular etliche Services und Features an, dazu gehören nicht nur die der Cross-Platform Support,
sondern auch eine built-in Geschwindigkeits- und Performanceoptimierung. Auch für Unternehmen, mit sehr hohen Nutzerzahlen
ist Angular ein gern gewähltes Tool. Ein großer Vorteil von Angular ist zudem, dass Angular unter eine Open-Source Lizenz veröffentlicht
ist und somit nicht von dem Angular-Google Team sondern auf von der Community betreut und erweitert wird.
Angular verfolgt bei der Implementierung das Konzept von der Model-View-Controller Architektur.

Angular zu verwenden ist eine Anforderung von der Betreuer Firma triply GmbH. Angular wird bereits für alle Frontend Projekte verwendet,
deshalb auch für dieses.

\subsection{Alternativen für Angular}
\setauthor{Jonas Dorfinger}

\subsubsection{React}
\setauthor{Jonas Dorfinger}
ReactJS ist die größte Konkurrenz von Angular, mit einem vielfachen an Marktanteilen.
ReactJS ist eine JavaScript Bibliothek von Facebook. Im Gegensatz zu Angular, wird der DOM weg abstrahiert was ein einfacheres 
Programmiermodel und bessere Performance ermöglicht. Ein Alleinstellungsmerkmal ist dazu auch noch, dass React Server-Side Rendering 
unterstützt.

\subsubsection{Vue}
\setauthor{Jonas Dorfinger}
VueJS ist kaum verbreitet, und hat als größten Nachteil, dass es nur von einer Person regelmäßig erweitert und betreut wird, das kann man anhand der Contributer Activity von dem 

% https://angular.io/guide/what-is-angular besucht am 09.02.2022
% https://angular.io/features besucht am 09.02.2022
% https://github.com/vuejs/core/graphs/contributors besucht am 09.02.2022

Alle 3 Frameworks werden von den den größten IDEs unterstützt.

\section{Map Frameworks}
\setauthor{Jonas Dorfinger}
Um die GeoDaten im Browser darstellen zu können, wird eine Map benötigt. Es gibt dafür drei weit verbreitete Frameworks, MapBox, Leaflet und Google Maps.

\subsection{Mapbox}
\subsection{Leaflet}
\subsection{Google Maps}

\section{Static Site Generators}
\setauthor{Sebastian Scholl}
\subsection{Next.js}
\subsection{Jekyll}
\subsection{Scully}

\section{CI/CD Pipeline}
\setauthor{Sebastian Scholl}
\subsection{Jenkins}
\subsection{GitHub Actions}

\section{Backend}
\setauthor{Sebastian Scholl}
\subsection{JavaScript}
\subsection{Typescript}

\section{Webserver}
\setauthor{Sebastian Scholl}

\section{Reverse Proxy}
\setauthor{Sebastian Scholl}

\section{Containerization}
\setauthor{Sebastian Scholl}
\subsection{Docker}
\subsection{Docker Compose}
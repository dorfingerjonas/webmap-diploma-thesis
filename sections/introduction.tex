\section{Kurzbeschreibung}
\setauthor{Jonas Dorfinger}
Triply entwickelt hochwertige Softwarelösungen, die dabei helfen, bestehende Mobilitätssituationen (Verkehrsanalysen,
Besucherströme) zu verstehen.
Eine der Kernfunktionen liegt darin, Ergebnisse der Analysen einfach und verständlich
darzustellen.
Dafür soll eine Software geschaffen werden, die interaktive Kunden-Demos einfach und schnell erzeugen kann.

\section{Aufgabenstellung}
\setauthor{Jonas Dorfinger}
Bei dem Diplomarbeitsprojekt webmap handelt es sich um die Implementierung eines Generators, welcher interaktive
Webseiten automatisch erstellt.
Für die erwartete Benutzung ist es erforderlich, dass die ganze Software vollständig im Webbrowser
und somit plattformunabhängig funktionieren wird.

\section{Zielsetzung}
\setauthor{Jonas Dorfinger}
Mitarbeiterinnen und Mitarbeitern von triply soll es mit der Datenvisualisierungs-Pipeline möglich sein, schnell und einfach
interaktive Websites auf der Basis von komplexen Datensätzen zu erstellen, um diese potentiellen Kundinnen und Kunden
anschaulich zu präsentieren.

\section{Geplantes Ergebnis}
\setauthor{Jonas Dorfinger}
Entwicklung eines effizienten Generators für interaktive Datenveranschaulichung.
Weiters ist eines der Ziele, die Benutzung der Software so leicht und intuitiv wie möglich zu gestalten.
Zusätzlich dazu soll der gesamte
Prozess vom Starten der Konfiguration bis zur laufenden Website möglichst wenig Zeit in Anspruch nehmen.
